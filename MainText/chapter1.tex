
\chapter{Introduction} \label{ch-1}

This research proposal is a significant extension of objectives and projects begun during my master's degree. Drinfeld modules have been a tool of theoretical interest in Number Theory since the 1970s when they were introduced by Valdimir Drinf'eld in his proof of the Langlands conjecture for the general linear group over a global function field of positive characteristic. Since then, there has been considerable interest in extending known results concerning Elliptic curves to the Drinfeld setting. This was partly the motivation of Gekeler in \cite{frobdist} who aimed to study the distribution of coefficients of the characteristic polynomials of Drinfeld modules to determine if an analog of the Sato-Tate conjecture might hold. Recent research has continued to scrutinize computational questions connected Drinfeld modules, which includes work that makes use of rank-two Drinfeld modules for factoring polynomials \cite{eschost2017arXiv171200669D}, as well as isogeny structures \cite{DBLP:books/ams/20/CaranayGS20} and endomorphism rings \cite{GaPa18}. Cryptographic primitives based on elliptic curves have been known for several decades and see use in real-world crypto-systems, however several attempts to extend these constructions to Drinfeld modules have proven insecure \cite{Scanlon2001PublicKC}, \cite{cryptoeprint:2019:1329}. 

To that end, the main goal of \cite{Musleh} was to develop an analog of Schoof's algorithm for counting points on an elliptic curve to the Drinfeld setting; while a direct parallel was not quite achieved, an algorithm in the spirit of Schoof's approach was developed, in addition to other algorithms that could leverage randomization for better practical performance. 

One of the main goals of our research has been to extend algorithms previously developed for the rank-two case to Drinfeld modules of arbitrary rank. One particular challenge in the higher rank case has been finding suitable classical methods on which to base our algorithms. While the relationship with elliptic curves is well established for the rank-two case, higher rank Drinfeld modules are less clearly analogized by classical algebro-geometric objects. 

Another major motivator for studying skew polynomials has been their use in coding theory, and in particular Gabidulin codes \cite{PUCHINGER2017b} \cite{bartz2021fast}. 




