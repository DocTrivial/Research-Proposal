\chapter{Objectives} \label{ch-2}

\section{Problem Statement}

The overarching research objective can be expressed with the following statement:

\begin{statement*}
The primary research subject of this thesis will be to study algorithms for operations on non-commutative algebraic structures. In particular, the following problems will be considered:

\begin{itemize}
    \item Computing the minimal polynomial of Drinfeld modules of rank $r > 2$.
    \item Multiplication algorithms for Ore polynomials where $\delta \neq 0$.
    \item 
\end{itemize}

\end{statement*}

There are three major outcomes of this thesis that we anticipate:

\begin{enumerate}
    \item Provide sharper complexity bounds for existing algorithms.

    \item Develop new algorithms related to non-commutative structures on finite fields that are competitive with the current theory.
    
    
    \item Develop concrete implementations of both new and existing algorithms to provide empirical verification of the theoretically derived complexities.
\end{enumerate}

With regards to point 1, algebraic complexity is the more widely used model for performing complexity analysis of algorithms over finite fields. This is related to another issue, namely that computing the action of a map on an element of a finite field is often assigned $O(1)$ complexity regardless of the cost of computing the map for the first time. To provide a more consistent framework within which to analyze algorithms, we aim to convert a number of these bounds to a bit-complexity model.


