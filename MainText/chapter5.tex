
\chapter{Further Directions} \label{ch-1}

There are a number of avenues of research we seek to pursue and include within the scope of the thesis. The most natural one being to continue our work on computing the minimal polynomial of the Frobenius endomorphism by extending other approaches, such as the use of the Hasse Invariant 


\section{Outcomes}

There are 3 key outcomes we wish to achieve:

\begin{enumerate}
    \item Devising new algorithms for elementary operations with various classes of skew polynomials
    \item Develop new algorithms for computations related to Drinfeld modules, and in particular computing the minimal polynomial.
    \item Create implementations for existing and new algorithms that achieve optimal theoretical bounds where this is feasible.
\end{enumerate}

With regards to item (1) there are a number of different directions that could be pursued. For example, skew rings where the commutation rule takes alternate forms such as $\tau a = (\sigma(a) + \delta)\tau$ for $a \in \L$ or which obey a Leibniz rule appear to be less well studied. 

