
\chapter{Further Directions} \label{ch-1}

There are a number of avenues of research we seek to pursue and include within the scope of the thesis. The most natural one being to continue our work on computing the minimal polynomial of the Frobenius endomorphism by extending other approaches, such as the use of the Hasse Invariant 


\section{Outcomes}

There are 3 key outcomes we wish to achieve:

\begin{enumerate}
    \item Devising new algorithms for elementary operations with various classes of skew polynomials. This is builds on both our previous efforts to provide a comprehensive analysis of operations related to skew polynomials, including multiplication, interpolation, and minimal subspace polynomial, in a bit complexity model.
    \item Develop new algorithms for computations related to Drinfeld modules, and in particular computing the minimal polynomial of Drinfeld modules of arbitrary rank $r > 2$. A significant amount of work in this regard has already been conducted and presented in chapter 4.
    \item Create implementations for existing and new algorithms that achieve optimal theoretical bounds where this is feasible. 
\end{enumerate}

With regards to item (1) there are a number of different directions that could be pursued. For example, skew rings where the commutation rule takes alternate forms such as $\tau a = (\sigma(a) + \delta)\tau$ for $a \in \L$ or which obey a Leibniz rule appear to be less well studied. However, inspired by the work on computing approximant bases of Bartz et al. in \cite{rosen2021}, we've identified multiplication of matrices with coefficients in a skew polynomial ring as a primary focus. In particular, a fast algorithm for computing minimal approximant basis is achieved by transporting the classical algorithm from the commutative case using a suitable bijective mapping. We wonder if a similar approach can be leveraged to construct a suitably fast algorithm for matrix multiplication over skew polynomial rings. 

As noted, a significant amount of work concerning the computation of minimal polynomials for higher rank Drinfeld modules has already carried out. However, we have identified two additional approaches that may yield further algorithms. The first is to generalize the Hasse invariant approach given by Gekeler in \cite{frobdist}. In the rank-two case when $\L = \F_{\frakp}$, computing the Frobenius trace reduces to computing the degree $m$ coefficient of $\phi_{\frakp}$, which is done using a recurrence based on Eisenstein series. analogs of this recurrence exist in higher ranks \cite{Gekeler1988}, and it may be possible to speed up the algorithm of Garai and Papikian using this approach. The second approach would be to attempt to generalize the algorithm of Kedlaya using $p$-adic cohmology to compute the zeta function for classical elliptic curves \cite{kedlaya01}. The initial goal with this approach would be to try to derive an algorithm that functions in the rank-two case at minimum, with a view towards extending this to all ranks.

With respect to item 3, some preliminary work including a number of basic operations and previous algorithms for computing the minimal polynomial of rank-two Drinfeld modules already exists and is publicly available at \url{https://github.com/ymusleh/Skew}. Our intention is to continue adding both existing algorithms and all algorithms developed as part of this thesis to the repository.
