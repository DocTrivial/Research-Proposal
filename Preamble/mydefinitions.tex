%----------------------------------------------------------------------------------------
%	VALUES FOR THE THESIS
%----------------------------------------------------------------------------------------

\newcommand{\name}{Joseph Yossef Musleh} % Author name
\newcommand{\thesistitle}{Algorithms for Rings of Skew-Polynomials and Drinfeld Modules} % Title of the thesis
\newcommand{\submissiondate}{October, 2019} % Submission date "Month, year"
\newcommand{\advisor}{\'Eric Schost} % Supervisor name
%\newcommand{\coadvisor}{Someone Someone} % Co-Supervisor name, comment this line if there is none


%----------------------------------------------------------------------------------------
%	BIBLIOGRAPHY STYLE (pick the style you want)
%----------------------------------------------------------------------------------------

\usepackage[square, numbers, sort&compress]{natbib} % for bibliography - Square brackets, citing references with numbers, citations sorted by appearance in the text and compressed (as in [4-7])
%\usepackage[longnamesfirst,round]{natbib} % Natural Sciences bibliography

\bibliographystyle{Preamble/physics_bibstyle} % You may use a different style adapted to your field
%\bibliographystyle{unsrtnat} % You may use a different style adapted to your field


%----------------------------------------------------------------------------------------
%	YOUR PACKAGES (be careful of package interaction)
%----------------------------------------------------------------------------------------

\usepackage{amsthm,amsmath,amssymb,amsfonts,bbm}% Math symbols

%----------------------------------------------------------------------------------------
%	YOUR DEFINITIONS AND COMMANDS
%----------------------------------------------------------------------------------------

% New Commands
\newcommand{\bea}{\begin{eqnarray}} % Shortcut for equation arrays
\newcommand{\eea}{\end{eqnarray}}
\newcommand{\e}[1]{\times 10^{#1}}  % Powers of 10 notation
\newcommand{\MSP}{\textnormal{MSP}}
\newcommand{\MPE}{\textnormal{MPE}}
\newcommand{\RED}{\textnormal{RD}}
\newcommand{\MSPr}{\overline{\textnormal{MSP}}}
\newcommand{\MPEr}{\overline{\textnormal{MPE}}}
\newcommand{\REDr}{\overline{\textnormal{RD}}}
\newcommand{\Mr}{\overline{\textnormal{M}}}
\newcommand{\skewdeg}{\textnormal{skew-deg}}
\newcommand{\id}{\textnormal{id}}
\newcommand{\spann}{\textnormal{span}}
%\newcommand{\frakp}{\mathfrak{p}}
%\usepackage{algorithm}
\newcommand{\SM}{\textnormal{M}}
\usepackage{algpseudocode}


% Defining a theorem box for Criteria
\newtheorem{critere}{Criterion}
\newcommand{\crit}[2]{
\begin{center}  
\fbox{ \begin{minipage}[c]{0.9 \textwidth}
\begin{critere}
\textbf{\textup{ #1}} --- #2
\end{critere}
\end{minipage}  } \end{center}
}

\newcommand\matr[1]
{%
  \ensuremath
  {
    \begin{bmatrix}
      #1
    \end{bmatrix}
  }
}